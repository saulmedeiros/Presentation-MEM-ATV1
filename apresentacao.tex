\documentclass[brazilian,svgnames,notes=hide,14pt, aspectratio=169]{beamer}
\mode<presentation>
\usepackage{macros}

\def\presentationtitle{Resolução Questão de Matemática Enem 2019}

\title{\presentationtitle}

\author[SILVA, S. M.]{Saulo Medeiros da Silva} 
    
\institute{Instituto Federal de Educação, Ciência e Tecnologia do Ceará IFCE - \textit{Campus} Cedro}

\date{\today}


\begin{document}


    \begin{frame}[plain, noframenumbering]
        \titlepage
    \end{frame}
    
    \begin{frame}[noframenumbering]
       \tableofcontents
    \end{frame}
    
    \begin{frame}{A questão}
    
        Iremos escolher uma questão de Matemática do Enem 2019. O número correspondente em cada caderno é:
        
            \begin{center}
                \circledg{161}\quad\circleda{166}\quad\circledam{171}\quad\circledp{177}
            \end{center}
    
    \end{frame}
    \section{A questão}
    \subsection{Enunciado}
    \begin{frame}{Enunciado da Questão}
        Construir figuras de diversos tipos, apenas dobrando e cortando papel, sem cola e sem tesoura, é a arte do origami (ori = dobrar; kami = papel), que tem um significado altamente simbólico no Japão.
        A base do origami é o conhecimento do mundo por base do tato. Uma jovem resolveu construir um cisne usando técnica do origami, utilizando uma folha de papel de \(18cm\) por \(12cm\).
Assim, começou por dobrar a folha conforme a figura.
    \end{frame}
    \subsection{Figura}
    \begin{frame}{Figura da Questão}
    
    \begin{center}
        
    
    \begin{tikzpicture} [scale=.25]
        \tkzDefPoint(0,0){E}
        \tkzDefPoint(12,0){C}
        \tkzDefPoint(12,12){B}
        \tkzDefPoint(-10,12){A}
        \tkzDefPoint(2.75,7){D}
    
        \tkzMarkRightAngle[fill=blue!20,size=1,draw](E,D,A)
        \tkzDrawSegment(E,D)
        \tkzDrawSegment(D,A)
        \tkzDrawPolygon(E,C,B,A)
        
        
        
        
        
        \tkzLabelSegment[font=\tiny](A,B){$18cm$}
        \tkzLabelSegment[font=\tiny, right](B,C){$12cm$}
        \tkzLabelSegment[font=\tiny, below](E,C){$12cm$}
        
        % \tkzDrawPoint(E)
        % \tkzDrawPoint(C)
        % \tkzDrawPoint(B)
        % \tkzDrawPoint(A)
        % \tkzDrawPoint(D)
        
        \tkzLabelPoints[font=\tiny, above](A,B,D)
        \tkzLabelPoints[font=\tiny, below](E,C)

\end{tikzpicture}
\end{center}
    \end{frame}
    \begin{frame}{Pergunta}
        Após essa primeira dobradura, a medida do segmento AE é:\\\vspace{.25cm}
        \circled{A}\xspace $2\sqrt{22}$\ cm\\
        \circled{B}\xspace $6\sqrt{3}$\ cm\\
        \circled{C}\xspace $12$\ cm\\
        \circled{D}\xspace $6\sqrt{5}$\ cm\\
        \circled{E}\xspace $2\sqrt{2}$\ cm\\
    \end{frame}
    
    \begin{frame}{Figura da Questão}
    
    \begin{center}
        
    
    \begin{tikzpicture} [scale=.25]
        \tkzDefPoint(0,0){E}
        \tkzDefPoint(12,0){C}
        \tkzDefPoint(12,12){B}
        \tkzDefPoint(-10,12){A}
        \tkzDefPoint(2.75,7){D}
    
        \tkzMarkRightAngle[fill=verdeif!20,size=1, draw](E,D,A)
        \tkzDrawSegment(E,D)
        \tkzDrawSegment(D,A)
        \tkzDrawPolygon(E,C,B,A)
        \tkzDrawSegment[color=redif](A,E)
        
        
        
        \tkzLabelSegment[font=\tiny](A,B){$18cm$}
        \tkzLabelSegment[font=\tiny, right](B,C){$12cm$}
        \tkzLabelSegment[font=\tiny, below](E,C){$12cm$}
        
         \tkzDrawPoint[color=redif, fill=redif](E)
        % \tkzDrawPoint(C)
        % \tkzDrawPoint(B)
         \tkzDrawPoint[color=redif, fill=redif](A)
        %  \tkzDrawPoint(D)
        
        \tkzLabelPoints[font=\tiny, above](A,B,D)
        \tkzLabelPoints[font=\tiny, below](E,C)

\end{tikzpicture}
\end{center}    
    \end{frame}
    \section{Entendendo a figura}
    \begin{frame}{Entendendo a figura}
    
    \begin{center}
        
    
    \begin{tikzpicture} [scale=.25]
        \tkzDefPoint(0,0){E}
        \tkzDefPoint(12,0){C}
        \tkzDefPoint(12,12){B}
        \tkzDefPoint(-10,12){A}
        \tkzDefPoint(2.75,7){D}
    
        \tkzMarkRightAngle[fill=verdeif!20,size=1, draw](E,D,A)
        \tkzDrawSegment[color=verdeif, dashed](E,D)
        \tkzDrawSegment(D,A)
        \tkzDrawSegment(B,C)
        \tkzDrawSegment(A,B)
        \tkzDrawSegment[color=redif](A,E)
        \tkzDrawSegment[dashed, color=verdeif](E,C)
        
        
        \tkzLabelSegment[font=\tiny](A,B){$18cm$}
        \tkzLabelSegment[font=\tiny, right](B,C){$12cm$}
        \tkzLabelSegment[font=\tiny, below](E,C){$12cm$}
        
         \tkzDrawPoint[color=redif, fill=redif](E)
        % \tkzDrawPoint(C)
        % \tkzDrawPoint(B)
         \tkzDrawPoint[color=redif, fill=redif](A)
        %  \tkzDrawPoint(D)
        
        \tkzLabelPoints[font=\tiny, above](A,B,D)
        \tkzLabelPoints[font=\tiny, below](E,C)

\end{tikzpicture}
\end{center}    
    \end{frame}
    \begin{frame}{Figura da Questão}
    
    \begin{center}
        
    
    \begin{tikzpicture} [scale=.25]
        \tkzDefPoint(0,0){E}
        \tkzDefPoint(12,0){C}
        \tkzDefPoint(12,12){B}
        \tkzDefPoint(-10,12){A}
        \tkzDefPoint(2.75,7){D}
    
        \tkzMarkRightAngle[fill=verdeif!20,size=1, draw](E,D,A)
        \tkzDrawSegment[color=verdeif, dashed](E,D)
        \tkzDrawSegment(D,A)
        \tkzDrawSegment(B,C)
        \tkzDrawSegment[color=verdeif, dashed](A,B)
        \tkzDrawSegment[color=redif](A,E)
        \tkzDrawSegment[dashed, color=verdeif](E,C)
        
        
        \tkzLabelSegment[font=\tiny](A,B){$18cm$}
        \tkzLabelSegment[font=\tiny, right](B,C){$12cm$}
        \tkzLabelSegment[font=\tiny, below](E,C){$12cm$}
        
         \tkzDrawPoint[color=redif, fill=redif](E)
        % \tkzDrawPoint(C)
        % \tkzDrawPoint(B)
         \tkzDrawPoint[color=redif, fill=redif](A)
        %  \tkzDrawPoint(D)
        
        \tkzLabelPoints[font=\tiny, above](A,B,D)
        \tkzLabelPoints[font=\tiny, below](E,C)

\end{tikzpicture}
\end{center}    
    \end{frame}
    \begin{frame}{Figura da Questão}
    
    \begin{center}
        
    
    \begin{tikzpicture} [scale=.25]
        \tkzDefPoint(0,0){E}
        \tkzDefPoint(12,0){C}
        \tkzDefPoint(12,12){B}
        \tkzDefPoint(-10,12){A}
        \tkzDefPoint(2.75,7){D}
    
        \tkzMarkRightAngle[fill=verdeif!20,size=1, draw](E,D,A)
        \tkzDrawSegment(E,D)
        \tkzDrawSegment(D,A)
        \tkzDrawSegment[dashed, color=azulenem](B,C)
        \tkzDrawSegment(A,B)
        \tkzDrawSegment[color=redif](A,E)
        \tkzDrawSegment(E,C)
        
        
        \tkzLabelSegment[font=\tiny](A,B){$18cm$}
        \tkzLabelSegment[font=\tiny, right](B,C){$12cm$}
        \tkzLabelSegment[font=\tiny, below](E,C){$12cm$}
        
         \tkzDrawPoint[color=redif, fill=redif](E)
        % \tkzDrawPoint(C)
        % \tkzDrawPoint(B)
         \tkzDrawPoint[color=redif, fill=redif](A)
        %  \tkzDrawPoint(D)
        
        \tkzLabelPoints[font=\tiny, above](A,B,D)
        \tkzLabelPoints[font=\tiny, below](E,C)

\end{tikzpicture}
\end{center}    
    \end{frame}
    \begin{frame}{Figura da Questão}
    
    \begin{center}
        
    
    \begin{tikzpicture} [scale=.25]
        \tkzDefPoint(0,0){E}
        \tkzDefPoint(12,0){C}
        \tkzDefPoint(12,12){B}
        \tkzDefPoint(-10,12){A}
        \tkzDefPoint(2.75,7){D}
    
        \tkzMarkRightAngle[fill=verdeif!20,size=1, draw](E,D,A)
        \tkzDrawSegment(E,D)
        \tkzDrawSegment[dashed, color=azulenem](D,A)
        \tkzDrawSegment[dashed, color=azulenem](B,C)
        \tkzDrawSegment(A,B)
        \tkzDrawSegment[color=redif](A,E)
        \tkzDrawSegment(E,C)
        
        
        \tkzLabelSegment[font=\tiny](A,B){$18cm$}
        \tkzLabelSegment[font=\tiny, right](B,C){$12cm$}
        \tkzLabelSegment[font=\tiny, below](E,C){$12cm$}
        
         \tkzDrawPoint[color=redif, fill=redif](E)
        % \tkzDrawPoint(C)
        % \tkzDrawPoint(B)
         \tkzDrawPoint[color=redif, fill=redif](A)
        %  \tkzDrawPoint(D)
        
        \tkzLabelPoints[font=\tiny, above](A,B,D)
        \tkzLabelPoints[font=\tiny, below](E,C)

\end{tikzpicture}
\end{center}    
    \end{frame}
    \section{Solução}
    \begin{frame}{Solução}
        \begin{minipage}{.5\textwidth}
            \begin{tikzpicture} [scale=.25]
                \tkzDefPoint(0,0){E}
                \tkzDefPoint(-10,12){A}
                \tkzDefPoint(2.75,7){D}
    
                
                \tkzDrawPolygon[fill=red!20](E,A,D)
                \tkzMarkRightAngle[fill=blue!20,size=1,draw](E,D,A)
                \tkzLabelSegment[font=\tiny, right](D,E){$6cm$}
                \tkzLabelSegment[font=\tiny, above](D,A){$12cm$}
        
                \tkzDrawPoint[fill=redif, color=redif](E)
                \tkzDrawPoint[fill=redif, color=redif](A)
                \tkzDrawPoint[fill=redif, color=redif](D)
                
                \tkzLabelPoints[font=\tiny, above](E,A,D)
                \tkzLabelSegment[left, font=\small](A,E){h}
        \end{tikzpicture}
    \end{minipage}
\begin{minipage}{.45\textwidth}
    \textbf{Teorema de Pitágoras}
    \small{\begin{gather*}
        a^2=b^2+c^2\\
       \invisible<1>{h^2=12^2+6^2}\\
       \invisible<1-2>{h^2=144+36}\\
        \invisible<1-3>{h^2=180}\\
        \invisible<1-4>{h=\sqrt{180}}\\
        \invisible<1-5>{h=\sqrt{6^2\cdot 5}}\\
        \invisible<1-6>{h=6\sqrt{5}}
        \invisible<1-7>{}
    \end{gather*}}
\end{minipage}
    \end{frame}
    \begin{frame}{Solução}
        Alternativa \circled{D}
    \end{frame}
    \begin{frame}
    \centering
        Obrigado!
    \end{frame}
\end{document}